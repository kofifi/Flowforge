\documentclass[aspectratio=169]{beamer}
\usepackage[utf8]{inputenc}
\usepackage[T1]{fontenc}
\usepackage{graphicx}
\usepackage{geometry}
\usepackage{caption}
\usepackage{booktabs}
\usepackage{hyperref}
\usepackage{lmodern}
\usepackage{tikz}
\usetikzlibrary{backgrounds,calc}

\usetheme{Madrid}
\useoutertheme{infolines}
\usecolortheme{seahorse}
\setbeamertemplate{navigation symbols}{}
\setbeamertemplate{footline}[frame number]
\setbeamertemplate{itemize items}[circle]
\setbeamersize{text margin left=8mm,text margin right=8mm}
% frametitle bez wypełnienia
\setbeamercolor{frametitle}{bg=}
\setbeamertemplate{frametitle}{%
  \vspace{2pt}%
  \begin{flushleft}\insertframetitle\end{flushleft}%
  \vspace{-4pt}%
}

% miękki, minimalistyczny wzór tła inspirowany Fluent Design
\definecolor{ffBase}{RGB}{244,245,241}   % --page-base
\definecolor{ffGreenSoft}{RGB}{201,225,212} % soft radial glow
\definecolor{ffGreenStrong}{RGB}{47,158,104} % --primary
\definecolor{ffGraphite}{RGB}{29,39,36}   % --primary-strong
\definecolor{ffBlueGlow}{RGB}{76,110,201} % dark glow used in theme-dark

\newcommand{\backgroundcanvas}{%
  \begin{tikzpicture}[remember picture,overlay]
    % bazowy panel w kolorystyce Flowforge
    \path [left color=ffBase,right color=white,opacity=0.9]
      (current page.south west) rectangle (current page.north east);
    % delikatne plamy koloru inspirowane tłem aplikacji
    \fill [ffGreenSoft,opacity=0.55] ($(current page.north east)+(-1.2,-0.8)$) circle (3.2);
    \fill [ffGreenSoft!80!white,opacity=0.45] ($(current page.north west)+(1.4,-1.1)$) circle (2.9);
    \fill [ffBlueGlow!30,opacity=0.32] ($(current page.south east)+(-1.6,1.4)$) circle (2.6);
    % subtelna siatka nawiązująca do canvasu react-flow
    \draw [ffGraphite!6,line width=0.25pt] ($(current page.south west)+(0,0)$) -- ($(current page.north east)+(0,0)$);
    \foreach \x in {0,...,12}{
      \draw [ffGraphite!5,line width=0.22pt] ($(current page.south west)+(0.75*\x,0)$) -- ($(current page.north east)+(0.75*\x,0)$);
      \draw [ffGraphite!5,line width=0.22pt] ($(current page.south west)+(0,0.75*\x)$) -- ($(current page.north east)+(0,0.75*\x)$);
    }
  \end{tikzpicture}%
}
\setbeamertemplate{background}{\backgroundcanvas}

\graphicspath{{light/}{dark/}}
\title{Flowforge UI \textendash{} przegląd widoków}
\author{Zespół Flowforge}
\date{Luty 2026}

% prosta komenda do wstawiania zrzutów z automatycznym skalowaniem w pionie
\newcommand{\shot}[2]{% filename, caption
  \begin{center}
    \includegraphics[height=0.7\textheight,keepaspectratio]{#1}\\[4pt]
    \parbox{0.9\linewidth}{\centering\small #2}
  \end{center}
}

\begin{document}

\begin{frame}
  \titlepage
\end{frame}

\begin{frame}{Agenda}
  \begin{itemize}
    \item Widoki główne (Workflows, Blocks, Executions, Scheduler)
    \item Edytor workflow (canvas, palette, variables)
    \item Detale wykonania (Execution details)
    \item Akcje na workflow (More \textrightarrow{} Versions/Rename/Import)
    \item Konfiguracja bloków (If, Switch, Calculation, HTTP itp.)
    \item Harmonogram (create/edit)
  \end{itemize}
\end{frame}

\section{Tryb jasny}

% --- Widoki główne ---
\begin{frame}{Workflows (light)}
  \shot{workflows.png}{Lista projektów workflow \textendash{} tryb jasny}
\end{frame}

\begin{frame}{Blocks (light)}
  \shot{blocks.png}{Widok bloków systemowych}
\end{frame}

\begin{frame}{Executions (light)}
  \shot{executions.png}{Historia wykonania skryptów}
\end{frame}

\begin{frame}{Scheduler (light)}
  \shot{scheduler.png}{Lista zaplanowanych zadań}
\end{frame}

% --- Edytor workflow ---
\begin{frame}{Workflow editor \textendash{} canvas}
  \shot{workflow-editor.png}{Początkowy widok edytora z blokami Start/End}
\end{frame}

\begin{frame}{Palette}
  \shot{workflow-editor-add-blocks.png}{Paleta dodawania bloków}
\end{frame}

\begin{frame}{Variables}
  \shot{workflow-editor-variables.png}{Lista zmiennych (dodawanie)}
\end{frame}

\begin{frame}{Variables \textendash{} edycja}
  \shot{workflow-editor-variables-edit.png}{Formularz edycji zmiennej w drawerze}
\end{frame}

% --- Execution details ---
\begin{frame}{Execution details}
  \shot{execution-details.png}{Pełne szczegóły przebiegu wykonania}
\end{frame}

% --- Akcje workflow ---
\begin{frame}{Workflow actions \textendash{} More}
  \shot{workflows-actions-01b-more-open.png}{Menu akcji \textquotedbl More\textquotedbl}
\end{frame}

\begin{frame}{Workflow actions \textendash{} Versions}
  \shot{workflows-actions-02-versions.png}{Lista wersji workflow}
\end{frame}

\begin{frame}{Workflow actions \textendash{} Rename}
  \shot{workflows-actions-03-rename-form.png}{Formularz zmiany nazwy}
\end{frame}

\begin{frame}{Workflow actions \textendash{} Import}
  \shot{workflows-actions-04-import.png}{Import workflow z pliku}
\end{frame}

% --- Konfiguracja bloków ---
\begin{frame}{If block}
  \shot{workflow-editor-config-if.png}{Konfiguracja bloku If}
\end{frame}

\begin{frame}{Switch block}
  \shot{workflow-editor-config-switch.png}{Konfiguracja bloku Switch}
\end{frame}

\begin{frame}{Calculation block}
  \shot{workflow-editor-config-calculation.png}{Konfiguracja bloku Calculation}
\end{frame}

\begin{frame}{Text Transform block}
  \shot{workflow-editor-config-texttransform.png}{Konfiguracja bloku Text Transform}
\end{frame}

\begin{frame}{Text Replace block}
  \shot{workflow-editor-config-textreplace.png}{Konfiguracja bloku Text Replace}
\end{frame}

\begin{frame}{HTTP request block}
  \shot{workflow-editor-config-httprequest.png}{Konfiguracja bloku HTTP request}
\end{frame}

\begin{frame}{Parser block}
  \shot{workflow-editor-config-parser.png}{Konfiguracja bloku Parser}
\end{frame}

\begin{frame}{Loop block}
  \shot{workflow-editor-config-loop.png}{Konfiguracja bloku Loop}
\end{frame}

\begin{frame}{Wait block}
  \shot{workflow-editor-config-wait.png}{Konfiguracja bloku Wait}
\end{frame}

% --- Run drawer (light) ---
\begin{frame}{Run \textendash{} panel danych (light)}
  \shot{light/run-drawer-open.png}{Panel Run z rozwiniętymi danymi wejściowymi i snippetem API}
\end{frame}

\begin{frame}{Run \textendash{} wynik uruchomienia (light)}
  \shot{light/run-drawer-result.png}{Panel Run po wykonaniu workflow \textendash{} widoczne wyniki i logi}
\end{frame}

% --- Scheduler actions ---
\begin{frame}{Scheduler action \textendash{} create}
  \shot{scheduler-actions-01-create.png}{Tworzenie wpisu schedulera}
\end{frame}

\begin{frame}{Scheduler action \textendash{} edit}
  \shot{scheduler-actions-02-edit.png}{Edycja istniejącego wpisu}
\end{frame}

\section{Tryb ciemny}

% --- Widoki główne ---
\begin{frame}{Workflows (dark)}
  \shot{dark/workflows-dark.png}{Lista projektów workflow \textendash{} tryb ciemny}
\end{frame}

\begin{frame}{Blocks (dark)}
  \shot{dark/blocks-dark.png}{Widok bloków systemowych \textendash{} tryb ciemny}
\end{frame}

\begin{frame}{Workflow editor (dark)}
  \shot{dark/workflow-editor.png}{Początkowy widok edytora \textendash{} tryb ciemny}
\end{frame}

\begin{frame}{Palette (dark)}
  \shot{dark/workflow-editor-add-blocks.png}{Paleta dodawania bloków \textendash{} tryb ciemny}
\end{frame}

\begin{frame}{Executions (dark)}
  \shot{dark/executions-dark.png}{Historia wykonania skryptów \textendash{} tryb ciemny}
\end{frame}

\begin{frame}{Scheduler (dark)}
  \shot{dark/scheduler-dark.png}{Lista zaplanowanych zadań \textendash{} tryb ciemny}
\end{frame}

% --- Akcje workflow ---
\begin{frame}{Workflow actions (dark) \textendash{} More}
  \shot{dark/workflows-actions-01b-more-open.png}{Menu akcji \textquotedbl More\textquotedbl{} (ciemny)}
\end{frame}

\begin{frame}{Workflow actions (dark) \textendash{} Versions}
  \shot{dark/workflows-actions-02-versions.png}{Lista wersji workflow (ciemny)}
\end{frame}

\begin{frame}{Workflow actions (dark) \textendash{} Rename}
  \shot{dark/workflows-actions-03-rename-form.png}{Formularz zmiany nazwy (ciemny)}
\end{frame}

\begin{frame}{Workflow actions (dark) \textendash{} Import}
  \shot{dark/workflows-actions-04-import.png}{Import workflow z pliku (ciemny)}
\end{frame}

% --- Execution details ---
\begin{frame}{Execution details (dark)}
  \shot{dark/execution-details.png}{Pełne szczegóły przebiegu \textendash{} tryb ciemny}
\end{frame}

% --- Konfiguracja bloków (dark) ---
\begin{frame}{If block (dark)}
  \shot{dark/workflow-editor-config-if.png}{Konfiguracja bloku If \textendash{} tryb ciemny}
\end{frame}

\begin{frame}{Switch block (dark)}
  \shot{dark/workflow-editor-config-switch.png}{Konfiguracja bloku Switch \textendash{} tryb ciemny}
\end{frame}

\begin{frame}{Loop block (dark)}
  \shot{dark/workflow-editor-config-loop.png}{Konfiguracja bloku Loop \textendash{} tryb ciemny}
\end{frame}

\begin{frame}{Wait block (dark)}
  \shot{dark/workflow-editor-config-wait.png}{Konfiguracja bloku Wait \textendash{} tryb ciemny}
\end{frame}

% --- Run drawer (dark) ---
\begin{frame}{Run \textendash{} panel danych (dark)}
  \shot{dark/run-drawer-open.png}{Panel Run z rozwiniętymi danymi wejściowymi i snippetem API}
\end{frame}

\begin{frame}{Run \textendash{} wynik uruchomienia (dark)}
  \shot{dark/run-drawer-result.png}{Panel Run po wykonaniu workflow \textendash{} widoczne wyniki i logi}
\end{frame}

\begin{frame}{Calculation block (dark)}
  \shot{dark/workflow-editor-config-calculation.png}{Konfiguracja bloku Calculation \textendash{} tryb ciemny}
\end{frame}

\begin{frame}{Text Transform block (dark)}
  \shot{dark/workflow-editor-config-texttransform.png}{Konfiguracja bloku Text Transform \textendash{} tryb ciemny}
\end{frame}

\begin{frame}{Text Replace block (dark)}
  \shot{dark/workflow-editor-config-textreplace.png}{Konfiguracja bloku Text Replace \textendash{} tryb ciemny}
\end{frame}

\begin{frame}{HTTP request block (dark)}
  \shot{dark/workflow-editor-config-httprequest.png}{Konfiguracja bloku HTTP request \textendash{} tryb ciemny}
\end{frame}

\begin{frame}{Parser block (dark)}
  \shot{dark/workflow-editor-config-parser.png}{Konfiguracja bloku Parser \textendash{} tryb ciemny}
\end{frame}

% --- Scheduler actions (dark) ---
\begin{frame}{Scheduler action (dark) \textendash{} create}
  \shot{dark/scheduler-actions-01-create.png}{Tworzenie wpisu schedulera \textendash{} tryb ciemny}
\end{frame}

\begin{frame}{Scheduler action (dark) \textendash{} edit}
  \shot{dark/scheduler-actions-02-edit.png}{Edycja istniejącego wpisu \textendash{} tryb ciemny}
\end{frame}

% --- Variables ---
\begin{frame}{Variables (dark) \textendash{} lista}
  \shot{dark/workflow-editor-variables.png}{Lista zmiennych (ciemny)}
\end{frame}

\begin{frame}{Variables (dark) \textendash{} edycja}
  \shot{dark/workflow-editor-variables-edit.png}{Edycja zmiennej (ciemny)}
\end{frame}

\begin{frame}{Zakończenie}
  \begin{itemize}
    \item Porządek slajdów: główne widoki $\rightarrow$ edytor $\rightarrow$ detale $\rightarrow$ konfiguracja bloków $\rightarrow$ scheduler
    \item Wszystkie zrzuty: katalog \texttt{tests/e2e/artifacts}
    \item W razie zmian w screenach: uaktualnij plik i skompiluj ponownie PDF
  \end{itemize}
\end{frame}

\end{document}
